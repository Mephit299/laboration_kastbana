%! Author = levi.hogdal
%! Date = 2023-09-19


% Preamble
\documentclass[11p]{article}
% Packages
\usepackage{amsmath}
\usepackage{graphicx}
\usepackage[swedish]{babel}
\usepackage[
    backend=biber,
    style=authoryear-ibid,
    sorting=ynt
]{biblatex}
\usepackage[utf8]{inputenc}
\usepackage[T1]{fontenc}
%Källor
\graphicspath{ {./images/} }

\title{Labrapprt \\ \small Fysik 1}
\author{Levi Högdal }
\date{\today}

\begin{document}

    \begin{titlepage}
        \begin{center}
            \vspace*{1cm}

            \Huge
            \textbf{Laboration 5}

            \vspace{0.5cm}
            \LARGE
            Ellära

            \vspace{1.5cm}

            \textbf{Ditt namn!}

            \vfill


            Fysik 1

            \vspace{0.8cm}

            \includegraphics[width=0.4\textwidth]{../images/NTI Gymnasiet_Symbol_print_svart.png}

            \Large
            Teknikprogrammet\\
            NTI Gymnasiet\\
            Umeå\\
            \today

        \end{center}
    \end{titlepage}
    \section{Syfte och frågeställning}
    \section{Del 1}
    Vi ska matematiskt bestämma med vart en kula kommer landa om man skjuter den ut ur en kulkanon och sen kolla om våra beräkningar stämmer med vart kulan hamnar i verklighet.
    \subsection{Material och metod}
    \paragraph{Material:}
    \begin{itemize}
        \item En kulkanon
        \item En kula
        \item En linjal/måtband
        \item En låda där kulan ska landa
    \end{itemize}
    \subsection{Metod}
    Först måste man bestämma utgångshastigheten av kulan.
    Man kan göra det genom att skjuta bollen rakt uppåt och kolla hur högt den kommer.
    När man vet hur högt bollen kommer kan man beräkna utgångshastigheten.
    När man vet utgångshastigheten ska man bestämmer vart bollen kommer landa när skillnaden i höjd för vart kullan sköts ifrån och landade är 0.
    Efter det ska man kolla hur långt kullan rör sig horisontellt om den landar på marken.

    \subsection{Resultat}
    När man vet hur högt bollen kommer kan man använda formeln $Y_{max}=(v^2 * sin^2(\alpha))/2g$ där $Y_{max}$ är hur högt bollen kom, g är gravitationskraft och $\alpha$ är $90^{\circ}$
    Formeln kan skrivas om till $v=\sqrt((2g*Y_{max})/sin^2(\alpha))$.
    \newline $v=\sqrt((2*9.82*1.29)/sin^2(60^{\circ})) = 5.03 m/s$
    \newline\newline
    För att sen lista ut vart kullan kommer landar kan man använde formeln $x_{max}=(v^2*sin(2\alpha))/g$
    Formeln antar att kullan kommer landa på samma höjd som den blev skjuten ifrån.
    För vårt experiment är v utgångshastigheten som beräknades tidigare, $\alpha$ är $60^{\circ}$ och g är gravitationskraft.
    \newline $x_{max}=(5.03^2*sin(2*60^{\circ}))/9.82 = 2.23 m$
    \newline\newline
    Efteråt skulle vi också kolla vart bollen skulle landa om man sköt den ner på golvet från bänken.
    För att veta vilken tid bollen träffade golvet.
    Man kan kolla tiden genom att kolla när y värdet för kullan är samma som golvet.
    Bänken var 73 cm högt.
    Man kan med ett grafritande verktyg kolla vart $5.03*sin(60)*x-(9.82*x^2)/2$ och $-0.73$
    När man gör det

    \begin{itemize}
        \item Utgångshastighet = $5.03 m/s$
        \item Hår långt kullan har farit horisontellt när den landade på bänken = 2.23 m
        \item Hår långt kullan har farit horisontellt när den landade på golvet = 2.59 m
    \end{itemize}
    \subsection{Analys}
    \section{Diskussion}
\end{document}
